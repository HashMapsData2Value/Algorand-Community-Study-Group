%----------------------------------------------------------------------------------------
%	PACKAGES AND THEMES
%----------------------------------------------------------------------------------------
\documentclass[aspectratio=169,xcolor=dvipsnames]{beamer}
\usetheme{Simple}

\usepackage{hyperref}
\usepackage{graphicx} % Allows including images
\usepackage{booktabs} % Allows the use of \toprule, \midrule and \bottomrule in tables
\usepackage{amssymb}
\usepackage{commath} % Align :=

%----------------------------------------------------------------------------------------
%	TITLE PAGE
%----------------------------------------------------------------------------------------

% The title
\title[short title]{Algorand Community Study Group}
\subtitle{2: L1s, L2s, and alternative blockchains}

\author[HMD2V] {HashMapsData2Value}
\institute[Algorand] % Your institution may be shorthand to save space
{
    % Your institution for the title page
    Digital Community Champion \\
    Algorand Foundation \\
    \vskip 3pt
}
\date{\today} % Date, can be changed to a custom date


%----------------------------------------------------------------------------------------
%	PRESENTATION SLIDES
%----------------------------------------------------------------------------------------

\begin{document}

\begin{frame}
    % Print the title page as the first slide
    \titlepage
\end{frame}


\begin{frame}{Welcome!}
\begin{itemize}
    \item Welcome!
    \item Study group? Discord: \href{https://discordapp.com/channels/491256308461207573/1092424979091566622}{\#cryptography-study-group}
    \item GitHub Repo: \href{https://github.com/HashMapsData2Value/Algorand-Community-Study-Group/tree/main/Topics}{https://github.com/HashMapsData2Value/Algorand-Community-Study-Group}
    \item Please introduce yourselves
    \item Topic - L1s, L2s, and alternative blockchains
\end{itemize}
\end{frame}

\begin{frame}{Agenda}
    % Throughout your presentation, if you choose to use \section{} and \subsection{} commands, these will automatically be printed on this slide as an overview of your presentation
    \tableofcontents[subsubsectionstyle=hide]
\end{frame}

\section{Layer 1}
\subsubsection{What makes up an L1?}
    \begin{frame}{Anatomy of an L1 - A network of nodes}
    \begin{itemize}
        \item Nodes
        \begin{itemize}
            \item Block assembly and validation software
            \item Networking layer, block propagation
            \item Blockchain history
        \end{itemize}
        \item Reach Consensus
        \begin{itemize}
            \item Who? Whose block...
            \item How? PoS, PoW, etc... 
            \item When? Block speed, finality...
        \end{itemize}
        \item Security Model
        \begin{itemize}
            \item Collusion and Nakamoto coefficient
            \item Cost of corruption
        \end{itemize}
        \item Abstraction: The World Computer
        \item Why need more than L1 in the first place?
    \end{itemize}

\end{frame}

\section{"Alternative" Blockchains}
\begin{frame}{Section 2}
    \Huge{\centerline{"Alternative" Blockchains}}
\end{frame}
\begin{frame}{Warning}
    \Huge{\centerline{These definitions are not 100\% set in stone.}}
\end{frame}
\subsubsection{Co-Chain}
\begin{frame}{Co-Chains}
    \begin{itemize} 
        \item Separate, basically copy+paste blockchain of an established L1, e.g. Algorand
        \item I.e. own and independent history, node network, security consideration
        \item Interoperability: Two-way bridge
        \item Permissioned vs Non-Permissioned, licensing
    \end{itemize}
\end{frame}

\subsubsection{Side-Chain}
\begin{frame}{Side-Chain}   
\begin{itemize}
    \item Separate and independent network of nodes, own security considerations
    \item BUT! Goal is to assist another L1 from the side
    \item Validators on side-chain may pay attention to "parent chain" (or vice-versa)
    \item E.g. Milkomeda C1 (Cardano EVM Sidechain):
    \begin{itemize}
        \item Token: mADA, wrap with Milkomeda Bridge
        \item Has a built-in EVM $ \rightarrow $ Solidity contracts
        \item Tx on Cardano calls contract on Milkomeda C1
    \end{itemize}
\end{itemize}
\end{frame}

\subsubsection{Oracles}
\begin{frame}{Oracles}
\begin{itemize}
    \item Goal: plug off-chain "truth" into on-chain
    \item Problem: Can't just trust one person's opinion/report
    \item Solution: "Ask the audience" - decentralized network of nodes, incentivized to reach consensus
    \item E.g. Goracle servicing Algorand:
    \begin{itemize}
        \item Token: GORA 
        \item Main (on-chain) contract:
        \begin{enumerate}
            \item Nodes stake their tokens, register public key
            \item Consumers request data, subscribe
        \end{enumerate}
        \item Off chain: Nodes poll main contract for requests, prepare data in block, PPoS sortition
        \item Votes for block tallied, data aggregated, presented to consumer 
        \item Summary: Nodes exist, but token and state is kept on-chain
    \end{itemize}
\end{itemize}    
\end{frame}
\subsubsection{Bridge Networks}
\begin{frame}{Bridge Networks}
\begin{itemize}
    \item Similar to Oracles - report activity on chain A to chain B
    \item Bridge components:
    \begin{itemize}
        \item "Deposit Box" contract on chain A, takes token X
        \item "Withdrawal Box" contract on chain B, gives wrapped token X
        \item Messaging service between the contract
    \end{itemize}
    \item How to prove to WB that you deposited X into DB so you can extract the wrapped tokens?
    \item $\rightarrow$ Rely on \textit{decentralized} network of nodes (with incentives)
    \item E.g., Portal/Wormhole (Messina.one), Flare
    \item (Better than "multi-sig"... Is there a better way?)
\end{itemize}
    
\end{frame}
\subsubsection{"Layer 0s"}
\begin{frame}{Layer 0s}
\begin{itemize}
    \item "Chain of Chains" approach
    \item Polkadot: Relay Chain + Parachains, bidding for slots
    \item Cosmos: Cosmos Hub + Zones, customized
    \item Avalanche: Primary Network Subnet: \textbf{C}ontract Chain + E\textbf{x}change Chain + \textbf{P}latform Chain, + custom subnets
\end{itemize}
\end{frame}

\section{Layer 2 solutions}
\begin{frame}{Section 3}
    \Huge{\centerline{Layer 2}}
\end{frame}
\subsubsection{Channels}
\begin{frame}{Channels}
\begin{itemize}
    \item Open: Lock up funds in multi-sig contract/account/tx, then construct and sign txs off-chain. Many txs can be passed back-and-forth, like ledger of IOUs.
    \item Close: Final up-to-date settlement tx is broadcast, distributing funds from contract on-chain.
    \item Multi-sig + timelock
    \item Simple (Payment), Advanced (State)
    \item E.g.: Bitcoin's Lightning Network
    \begin{itemize}
        \item Network of bidirectional payment channels
        \item Onion routing
        \item Cooperative vs Uncooperative Closing
        \item Dispute period $\rightarrow$ Publish most up-to-date tx
    \end{itemize}
\end{itemize}
\end{frame}

\subsubsection{Rollups}
\begin{frame}{Rollups}
\begin{itemize}
    \item Rather than 100 people signing 100 tx requiring 100 signature checks on L1, what if they were all bundled up?
    \begin{enumerate}
        \item Send tx to rollup "sequencer" or group of "operators"
        \item Sequencer pools tx batch and posts to L1
        \item Rollup "validators" validate batch off-chain
        \item IF valid, validators post rollup block to L1 with proof
    \end{enumerate}
    \item \textit{Crucially}, that proof can be verified \textit{within} the L1! And, transaction data got posted.
    \item After requesting withdrawal, user posts merkle proof validating inclusion of user's withdrawal tx in verified tx batch.
    \item Types of proof?
    \begin{itemize}
        \item Optimistic Rollups: Fraud Proofs
        \item ZK Rollups: Validity Proofs
    \end{itemize}
\end{itemize}
\end{frame}
\begin{frame}{Optimistic Rollups}
    \begin{itemize}
        \item We assume \textit{optimistically} that validators are acting honestly/non-maliciously when posting rollup block. They don't need to \textit{prove} anything, just provide stake.
        \item If malicious rollup block is posted, as long as there's at least 1 honest node auditing, a fraud proof will be provided as a challenge, the rollup block rejected and the stake lost to that challenger. 
        \item "Challenge time" means long settlement times! To give adequate time for challengers even with some malicious nodes. One week wait to withdraw funds from L2 contract.
        \item Optimistic Rollups can incorporate many EVM specs, allowing for sophisticated EVM contracts to be deployed.
    \end{itemize}
\end{frame}
\begin{frame}{ZK Rollups}
    \begin{itemize}
        \item Rather than optimistic, pessimistic and suspicious. Requires Zero-Knowledge Proofs of validity, which have their own considerations.
        \item Allows for very fast finality.
        \item ZK Rollups good for rolling up simple txs, research topic how to translate EVM opcodes over.
        \item Considered the "future" of rollups and scaling on e.g. Ethereum.
    \end{itemize}
\end{frame}
\subsubsection{Plasma Chain}
\begin{frame}{Plasma Chain}
\begin{itemize}
    \item Separate blockchain "anchored" to a "root" chain, but executing tx off-chain with own mechanism for block validation. Similar to side-chains but benefit from main chain security.
    \item Most often just one validation node, settling off-chain then posting merkle root "state commitments" to contract on Ethereum. Contract processes user fund entry/exit.
    \item Plasma contract accepts \textbf{fraud proofs}, like optimistic rollups, but stores the transaction data off-chain.
\end{itemize}
\end{frame}
\subsubsection{Validium}
\begin{frame}{Validium}
\begin{itemize}
    \item Validiums, like ZK Rollups, post ZKPs (ZK-SNARKs or ZK-STARKs) to a contract on Ethereum. This allows near-instant withdrawals (once ZKP is generated!).
    \item Unlike ZK Rollups HOWEVER validium contracts do NOT keep tx data on-chain, instead off-chain, saving on cost. If data availability managers withhold off-chain state data from users, users cannot withdraw.
    \item E.g. StarkWare StarkEx runs in both ZK Rollup or Validium data-availability modes.
\end{itemize}
\end{frame}

\subsubsection{Future of Ethereum}
\begin{frame}{Future of Ethereum}
\begin{itemize}
    \item \textit{\href{https://ethereum-magicians.org/t/a-rollup-centric-ethereum-roadmap/4698/1}{A Rollup-centric Ethereum Roadmap}} - Vitalik Buterin's forum post
    \item "[...] the Ethereum ecosystem is likely to be all-in on rollups (plus some plasma and channels) as a scaling strategy for the near and mid-term future."
    \item Dank-sharding: "instead of providing more space for transactions, Ethereum sharding provides more space for blobs of data [Ethereum] itself does not attempt to interpret. Verifying a blob simply requires checking that the blob is available - that it can be downloaded from the network. The data space in these blobs"
\end{itemize}
\end{frame}
\subsubsection{Comparison Table}
\begin{frame}{Comparison Table}
\begin{tabular}{|c|c|c|c|}
\hline
    L2 Name & On-Chain Data? & Fraud Proof? & Validity Proof? \\ 
\hline
    Channel &  No & No & No \\
\hline
    Optimistic Rollup & Yes & Yes & No \\
\hline
    ZK Rollup & Yes & No & Yes \\
\hline
    Plasma & No & Yes & No \\ 
\hline
    Validium & No & No & Yes \\
\hline
\end{tabular}
\end{frame}



\section{Algorand}
\begin{frame}{Section 4}
    \Huge{\centerline{Algorand}}
\end{frame}
\subsubsection{Algorand - pros & cons}
\begin{frame}{Algorand - pros & cons}
\begin{itemize}
    \item Algorand has \textbf{instant settlement finality}, soon 3.3s blocktime and 10k+ TPS 
    \item PPoS leader selection $\rightarrow$ bottle-neck lies in networking
    \item Where can L2 come in?
    \begin{itemize}
        \item Smart contracts limited by opcode budget
        \item Application specific L2? (If co-chain "too much".)
    \end{itemize}
    \item IMO the real improvements relate to relay nodes and archiving nodes...
\end{itemize}
\end{frame}

\subsubsection{Algorand Inc's L2 Research}
\begin{frame}{Algorand Inc's L2 Research}
\begin{itemize}
    \item \href{https://algorand.com/resources/blog/algorand-smart-contract-architecture}{Algorand’s Smart Contract Architecture - "Our Two Tier Architecture", Blog post by Silvio (May 2020)}
    \item \href{https://www.youtube.com/watch?v=fIRu0FvEUwQ}{Speculative Smart Contracts in the Algorand Blockchain - Victor Luchangco (June 2022)}
    \item Problem: Accomodating data-intensive and computation-intensive esoteric contracts 
    \begin{itemize}
    \item Clarity lang, a "decidable, predictable" language. Smart contracts kept in L2 off-chain but called from L1.
    \item L2 nodes running PPoS run  batch result of contract calls and collect signatures in compact certificates (state proofs) which can verified in AVM on L1.
    \end{itemize}
    \item For now L1's AVM been able to grow very well. No recent update (AFAIK).
\end{itemize}
\end{frame}

\subsubsection{Milkomeda AVM Rollup}
\begin{frame}{Milkomeda AVM Rollup}
\begin{itemize}
    \item Brings EVM to Algorand. Wrap Algo $\rightarrow$ mAlgo on Milkomeda A1, with bridge
    \item Still in development, not yet using BoxStorage but combined tx notes of 16 group tx (16 kb).
\end{itemize}
\begin{enumerate}
    \item Batch Proposal Phase: Sequencer divides batch of txs into groups, create Merkle root of batch. Post Merkle root on-chain to declare "Block Proposal" and initiate next phase.
    \item Batch Availability Phase: With Merkle root on-chain (in "block 0"), all batch data must be posted within 6 blocks. Tx groups + Merkle proof. If a new proposal is posted it'll be ignored, or if not all the data is posted within 6 blocks it'll be considered invalid. Rollup nodes observe everything, verify the Merkle tree and that the batch is "available".
    \item Batch Validation Phase: Locally, rollup nodes verify that the transactions (which could be EVM calls) are valid. If they are, the state of the rollup advances a step. Otherwise they do not and state moves back into Batch Proposal.
\end{enumerate}
\end{frame}
\begin{frame}{Milkomeda AVM Rollup}
\begin{itemize}
    \item Who gets to be sequencer? Who gets to be validator?
    \begin{itemize}
        \item Sequencer and Validator elections using auction to allocate "slots" of blocks on L1 to be sequencer
        \item Proof of stake with stake locked up in rollup contract, VRF using L1 for randomness seed
        \item Reward of mAlgo if honest, slashing if dishonest (can tolerate 1/3 malicious)
        \item Currently in "basic" mode, relying on single centralized sequencer/batcher
    \end{itemize}
    \item How to exit? Provide an exit tx signed by 2/3 of the node validators to contract.
    \item Problem: Algorand cannot verify EVM operations. Thus the L1 cannot be relied on the validate state of rollup.
    \begin{itemize}
        \item On-going problem: "The current goal to be implement for the Algorand EVM Rollup is a fraud proof system." \href{https://docs.milkomeda.com/algorand/how-it-works/optimistic-rollup}{Optimistic rollup}, page still under construction.
    \end{itemize}
    \item On-going research into BoxStorage
\end{itemize}
\end{frame}

\subsubsection{Algorand as Data Availability Layer}
\begin{frame}{Algorand as Data Availability Layer}
\begin{itemize}
    \item \href{https://twitter.com/SebastienGllmt/status/1619451660102938627?s=20}{Tweet by Sebastien Guillemot} (DcSpark, Milkomeda, PaimaStudios) 
    \item \href{https://github.com/algorand/go-algorand/pull/4924}{AVM EC Math upgrade} - coming soon
    \item Problem: Storing data off-chain AND verifying it is "still there".
\end{itemize}    
Sebastien suggest the following:
\begin{enumerate}
    \item Store data in BoxStorage from other chain
    \item Create (off-chain) KZG commitments of data in BoxStorage
    \item Wait for state proof, it will verify the successful inclusion of BoxStorage tx
    \item Take state proof and into other chain
    \item Some time later...
    \item Request comes to validate data is this available
    \item Trigger data availability sampling (randomly sample and check it exists in various places), leveraging KZG commitment scheme
    \item Validate KZG commitment (pairing check) inside AVM
    \item Take state proof into other chain
\end{enumerate}
\end{frame}


\subsubsection{Mixers as L2s?}
\begin{frame}{Mixers as L2s?}
\begin{itemize}
    \item Are mixers L2s? Mixers are a type of "one-shot" L2s.
    \item Mixer: smart contract that you (along with others) deposit a fixed chunk X of money into from account A. Later, you withdraw that money from account B, using a proof that you are \textit{one of} the depositors, without revealing exactly \textit{who}. Your money is mixed up and obfuscated, granting privacy through plausible deniability.
    \item When depositing you provide a "commitment" or public key; to withdraw you provide a "nullifier" or "key image". (Merkle Proof ZKP vs Ring Signatures.)
    \item Wallet Layers:
    \begin{enumerate}
        \item First layer is account-based Algorand. You maintain your keys for accounts A and B. 
        \item Second layer is UTXO-based mixer contract. E.g. 20 000 Algo in 100 Algo increments requires storing 200 withdrawal proofs, allowing 200 transactions.
    \end{enumerate}
    \end{itemize}
\end{frame}

\subsubsection{Open discussion}
\begin{frame}{Open discussion}
\begin{itemize}
    \Huge{\centerline{Any other thoughts?}}
\end{itemize}
\end{frame}


%-------------------------------------------- TEMPLATE!




%------------------------------------------------

\begin{frame}
    \Huge{\centerline{The End (for now!)}}
\end{frame}

%----------------------------------------------------------------------------------------

\end{document}